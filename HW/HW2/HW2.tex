\documentclass{article}

% Formatting
\usepackage[utf8]{inputenc}
\usepackage[margin=1in]{geometry}
\usepackage[titletoc,title]{appendix}

% Math
\usepackage{amsmath,amsfonts,amssymb,mathtools}

% Algorithms
\usepackage[ruled,vlined]{algorithm2e}
\usepackage{algorithmic}

% Code syntax highlighting
\usepackage{minted}
\usemintedstyle{borland}

\title{Homework 2}
\author{David Trinh}
\date{September 24, 2024}

\begin{document}

\maketitle

\begin{itemize}

    \item\textbf{ Question 1}
    
    \textbf{(a)} $f(n) = n^2 + 3n + 2, f(n) = O(n^2)$

    Let positive constants $c$ and $n_o$, we have:

    \begin{center}
        $n^2 + 3n + 2 \le c\cdot n^2$ for all $n \ge n_0$
    \end{center}

    $$1 + \frac{3}{n} + \frac{2}{n^2} \le c$$

    Let $n$ be 1, we have:

    $$1 + \frac{3}{1} + \frac{2}{1^2} \le c$$

    $$6 \le c$$

    As $n \to \infty$, the terms $\frac{3}{n}$ and $\frac{2}{n^2}$ tend to $0$.

    Thus, for all $n \ge 1$, $c \ge 6$.

    Therefore, there exist $n_0 = 1$ and $c = 6$.\\

    \textbf{(b)} $f(n) = 4n^3 + n^2 + nlogn + 5, f(n) = \Theta(n^3)$
    
    Let positive constants $c_1$, $c_2$, and $n_o$, we have:

    \begin{center}
        $c_1\cdot n^3 \le 4n^3 + n^2 + nlogn + 5 \le c_2\cdot n^3$ for all $n \ge n_0$
    \end{center}

    $$c_1 \le 4 + \frac{1}{n} + \frac{logn}{n^2} + \frac{5}{n^3} \le c_2$$

    Let $n$ be 1, we have:

    $$c_1 \le 4 + \frac{1}{1} + \frac{log1}{1^2} + \frac{5}{1^3} \le c_2$$

    $$c_1 \le 10 \le c_2$$

    As $n \to \infty$, the terms $\frac{1}{n}$, $\frac{logn}{n^2}$, and $\frac{5}{n^3}$ tend to $0$.

    Thus, for all $n \ge 1$, $c_1 \le 10 \le c_2$.

    Therefore, there exist $n_0 = 1$ and $c = 10$.\\

    \textbf{(c)} $f(n) = n^2 - 8n + 1, f(n) = \Omega(n)$

    Let positive constants $c$ and $n_o$, we have:

    \begin{center}
        $n^2 - 8n + 1 \ge c\cdot n$ for all $n \ge n_0$
    \end{center}

    $$n - 8 + \frac{1}{n} \ge c$$

    Let $n$ be 9, we have:

    $$9 - 8 + \frac{1}{9} \ge c$$

    $$\frac{10}{9} \ge c$$

    As $n \to \infty$, the term $n$ tends to $\infty$ and $\frac{1}{n}$ tends to $0$.

    Thus, for all $n \ge 9$, $c \le \frac{10}{9}$.

    Therefore, there exist $n_0 = 9$ and $c = 1$.\\


    \item\textbf{ Question 2}
        \begin{equation}
            r=\sum^{n}_{i=1} \sum^{i}_{j=1} \sum^{i+j}_{k=j} 1
        \end{equation}
        \begin{equation}
            r=\sum^{n}_{i=1} \sum^{i}_{j=1} (i+1)
        \end{equation}
        \begin{equation}
            r=\sum^{n}_{i=1} i(i+1)
        \end{equation}
        \begin{equation}
            r=\sum^{n}_{i=1} i^2 + \sum^{n}_{i=1} i
        \end{equation}
        \begin{equation}
            r=\frac{n(n+1)(2n+1)}{6} + \frac{n(n+1)}{2}
        \end{equation}
        \begin{equation}
            r=\frac{(n^2+n)(2n+1) + 3(n^2+n)}{6}
        \end{equation}
        \begin{equation}
            r=\frac{2n^3 + n^2 + 2n^2 + n + 3n^2 + 3n}{6}
        \end{equation}
        \begin{equation}
            r=\frac{n^3 + 3n^2 + 2n}{3}
        \end{equation}
        \begin{itemize}
            \item[] Big-O: $O(n^3)$
            \item[] Big-$\Omega$: $\Omega(n^3)$
            \item[] Big-$\Theta$: $\Theta(n^3)$
        \end{itemize}
\end{itemize}

\end{document}